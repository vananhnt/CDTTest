\chapter {Maintaining {\REDUCE}}

Since January 1, 2009 {\REDUCE} is Open Source Software. It is hosted at
\begin{quote}
  \url{http://reduce-algebra.sourceforge.net/}
\end{quote}
We mention here three ways in which {\REDUCE} is maintained. The first is
the collection of queries, observations and bug-reports. All users are
encouraged to subscribe to the
\href{https://sourceforge.net/mail/?group_id=248416}{mailing list}
that Sourceforge.net provides so that they will receive information
about updates and concerns. Also on SourceForge there is a
\href{https://sourceforge.net/tracker/?group_id=248416}{bug tracker}
and a \href{https://sourceforge.net/forum/?group_id=248416}{forum}.
The expectation is that the maintainers and keen users of {\REDUCE} will
monitor those and try to respond to issues. However these resources
are not there to seek answers to Maths homework problems - they are
intended specifically for issues to do with the use and support of
{\REDUCE}.

The second level of support is provided by the fact that all the sources of 
{\REDUCE} are available, so any user who is having difficulty either with a bug 
or understanding system behaviour can consult the code to see if (for 
instance) comments in it clarify something that was unclear from the regular 
documentation.

The source files for {\REDUCE} are available on SourceForge in the
\href{https://sourceforge.net/svn/?group_id=248416}{Subversion
  repository}. Check the "code/SVN" tab on the SourceForge page to
find instructions for using a Subversion client to fetch the most up
to date copy of everything. From time to time there may be one-file
archives of a snapshot of the sources placed in the download area on
SourceForge, and eventually some of these mat be marked as ``stable''
releases, but at present it is recommended that developers use a copy
from the Subversion repository.

The files fetched there come with a directory called ``trunk'' that holds the 
main current {\REDUCE}, and one called ``branches'' that is reserved for future 
experimental versions. All the files that we have for creating help files and 
manuals should also be present in the files you fetch.

The packages that make up the source for the algebraic capabilities of
{\REDUCE} are in the ``packages'' sub-directory, and often there are test
files for a package present there and especially for contributed
packages there will be documentation in the form of a \LaTeX{} file.
Although {\REDUCE} is coded in its own language many people in the past
have found that it does not take too long to start to get used to it.

In various cases even fairly ``ordinary end users'' may wish to fetch the 
source version of {\REDUCE} and compile it all for themselves. This may either 
be because they need the benefit of a bug-fix only recently checked into the 
subversion repository or because no pre-compiled binary is available for the 
particular computer and operating system they use. This latter is to some 
extent unavoidable since {\REDUCE} can run on both 32 and 64-bit Windows, the 
various MacOSX options (eg Intel and Powerpc), many different distributions 
of Linux, some BSD variants and Solaris (at least). It is not practically 
feasible for us to provide a constant stream of up to date ready-built 
binaries for all these.

There are instructions for compiling {\REDUCE} present at the top of the trunk 
source tree. Usually the hardest issue seems to be encuring that your 
computer has an adequate set of development tools and libraries installled 
before you start, but once that is sorted out the hope is that the 
compilation of {\REDUCE} should proceed uneventfully if sometimes tediously.

In a typical Open Source way the hope is that some of those who build {\REDUCE} 
from source or explore the source (out of general interest or to pursue an 
understanding of some bug or detail) will transform themselves into 
contributors or developers which moves on to the third level of support.

At this third level any user can contribute proposals for bug fixes or
extensions to {\REDUCE} or its documentation. It might be valuable to
collect a library of additional user-contributed examples illustrating
the use of the system too. To do this first ensure that you have a
fully up to date copy of the sources from Subversion, and then
depending on just what sort of change is being proposed provide the
updates to the developers via the SourceForge bug tracker or other
route. In time we may give more concrete guidance about the format of
changes that will be easiest to handle. It is obviously important that
proposed changes have been properly tested and that they are
accompanied with a clear explanation of why they are of benefit. A
specific concern here is that in the past fixes to a bug in one part
of {\REDUCE} have had bad effects on some other applications and
packages, so some degree of caution is called for. Anybody who
develops a significant whole new package for {\REDUCE} is encouraged to
make the developers aware so that it can be considered for inclusion.

So the short form explanation about Support and Maintenance is that it is 
mainly focussed around the SourceForge system. That if discussions about 
bugs, requirements or issues are conducted there then all users and potential 
users of {\REDUCE} will be able to benefit from reviewing them, and the 
Sourceforge mailing lists, tracker, forums and wiki will grow to be both a 
static repository of answers to common questions, an active set of locations 
to to get new issues looked at and a focus for guiding future development.


