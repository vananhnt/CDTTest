PM is a general pattern matcher similar in style to those found in systems such
as SMP and Mathematica, and is based on the pattern matcher described in Kevin
McIsaac, ``Pattern Matching Algebraic Identities'', SIGSAM Bulletin, 19 (1985),
4-13.  The following is a description of its structure.

A \emph{template} is any expression composed of literal elements,
e.g. \texttt{5}, \texttt{a}, or \texttt{a+1}, and specially-denoted pattern
variables, e.g. \texttt{?a} or \texttt{??b}.  Atoms beginning with \texttt{?}
are called \emph{generic variables} and match any expression.

Atoms beginning with \texttt{??} are called \emph{multi-generic variables} and
match any expression or any sequence of expressions including the null or empty
sequence.  A sequence is an expression of the form \texttt{[a1,a2,...]}.  When
placed in a function argument list the brackets are removed,
i.e. \texttt{f([a,1]) -> f(a,1)} and \texttt{f(a,[1,2],b) -> f(a,1,2,b)}.

A template is said to match an expression if the template is literally equal to
the expression, or if by replacing any of the generic or multi-generic symbols
occurring in the template, the template can be made to be literally equal to the
expression.  These replacements are called the \emph{bindings} for the generic
variables.  A replacement is an expression of the form \texttt{exp1 -> exp2},
which means \texttt{exp1} is replaced by \texttt{exp2}, or \texttt{exp1 -->
exp2}, which is the same except \texttt{exp2} is not simplified until after the
substitution for \texttt{exp1} is made.  If the expression has any of the
properties associativity, commutativity, or an identity element, they are used
to determine if the expressions match.  If an attempt to match the template to
the expression fails the matcher backtracks, unbinding generic variables, until
it reaches a place where it can make a different choice.  It then proceeds along
the new branch.

The current matcher proceeds from left to right in a depth-first search of the
template expression tree.  Rearrangements of the expression are generated when
the match fails and the matcher backtracks.

The matcher also supports \emph{semantic} matching.  Briefly, if a subtemplate
does not match the corresponding subexpression because they have different
structures, then the two are equated and the matcher continues matching the rest
of the expression until all the generic variables in the subexpression are
bound.  The equality is then checked.  This is controlled by the
switch \texttt{semantic}.  By default it is \texttt{on}.

\subsection{\texttt{M(exp,temp)}} 

The template \texttt{temp} is matched against the expression \texttt{exp}.  If
the template is literally equal to the expression \texttt{T} is returned.  If
the template is literally equal to the expression after replacing the generic
variables by their bindings then the set of bindings is returned as a set of
replacements.  Otherwise \texttt{0} (\texttt{nil}) is returned.

\subsubsection*{Examples:}

A ``literal'' template:
\begin{verbatim}
        m(f(a), f(a));
        t
\end{verbatim}
Not literally equal:
\begin{verbatim}
        m(f(a), f(b));
        0
\end{verbatim}
Nested operators:
\begin{verbatim}
        m(f(a,h(b)), f(a,h(b)));
        t
\end{verbatim}
``Generic'' templates:
\begin{verbatim}
        m(f(a,b), f(a,?a));
        {?a -> b}
        m(f(a,b), f(?a,?b));
        {?b -> b, ?a -> a}
\end{verbatim}
The multi-generic symbol \texttt{??a} matches the ``rest'' of the arguments:
\begin{verbatim}
        m(f(a,b), f(??a));
        {??a -> {[a, b]}
\end{verbatim}
but the generic symbol \texttt{?a} does not:
\begin{verbatim}
        m(f(a,b), f(?a));
        0
\end{verbatim}
Flag $h$ as ``associative'':
\begin{verbatim}
        flag('(h), 'assoc);
\end{verbatim}
Associativity is used to group terms together:
\begin{verbatim}
        m(h(a,b,d,e), h(?a,d,?b));
        {?b -> e, ?a -> h(a,b)}
\end{verbatim}
``plus'' is a symmetric function:
\begin{verbatim}
        m(a+b+c, c+?a+?b);
        {?b -> a, ?a -> b}
\end{verbatim}
and it is also associative
\begin{verbatim}
        m(a+b+c, b+?a);
        {?a -> c + a}
\end{verbatim}
Note that the effect of using a multi-generic symbol is different:
\begin{verbatim}
        m(a+b+c,b+??c);
        {??c -> [c,a]}
\end{verbatim}



\subsection{temp \_= logical\_exp} 

A template may be qualified by the use of the conditional operator \texttt{\_=},
\texttt{such!-that}.  When a \texttt{such-that} condition is encountered in a template, it
is held until all generic variables appearing in \texttt{logical\_exp} are bound.

On the binding of the last generic variable, \texttt{logical\_exp} is simplified
and if the result is not \texttt{T} the condition fails and the pattern matcher
backtracks.  When the template has been fully parsed any remaining
held \texttt{such-that} conditions are evaluated and compared to \texttt{T}.

\subsubsection*{Examples:}

\begin{verbatim}
        m(f(a,b), f(?a,?b\_=(?a=?b)));
        0
        m(f(a,a), f(?a,?b\_=(?a=?b)));
        {?b -> a, ?a -> a}
\end{verbatim}
Note that \texttt{f(?a,?b\_=(?a=?b))} is the same as \texttt{f(?a,?a)}.


\subsection[S(exp,\{temp1 -> sub1, temp2 -> sub2, \ldots\}, rept, depth)]{S(exp,\{temp1 $\to$ sub1, temp2 $\to$ sub2, \ldots\}, rept, depth)}

Substitute the set of replacements into \texttt{exp}, re-substituting a maximum
of \texttt{rept} times and to a maximum depth \texttt{depth}. \texttt{rept} and
\texttt{depth} have the default values of 1 and $\infty$ respectively.
Essentially, \texttt{S} is a breadth-first search-and-replace.  (There is also a
depth-first version, \texttt{Sd(...)}.)  Each template is matched against
\texttt{exp} until a successful match occurs.

Any replacements for generic variables are applied to the r.h.s. of that
replacement and \texttt{exp} is replaced by the r.h.s.  The substitution process
is restarted on the new expression starting with the first replacement.  If none
of the templates match \texttt{exp} then the first replacement is tried against
each sub-expression of \texttt{exp}. If a matching template is found then the
sub-expression is replaced and process continues with the next sub-expression.

When all sub-expressions have been examined, if a match was found, the
expression is evaluated and the process is restarted on the sub-expressions of
the resulting expression, starting with the first replacement.  When all
sub-expressions have been examined and no match found the sub-expressions are
reexamined using the next replacement.  Finally when this has been done for all
replacements and no match found then the process recures on each sub-expression.
The process is terminated after \texttt{rept} replacements or when the
expression no longer changes.

The command
\begin{verbatim}
        Si(exp, {temp1 -> sub1, temp2 -> sub2, ...}, depth)
\end{verbatim}
means ``substitute infinitely many times until expression stops changing''.  It
is short-hand for \texttt{S(exp,\{temp1 -> sub1, temp2 -> sub2,...\},Inf, depth)}.


\subsubsection*{Examples:}

\begin{verbatim}
        s(f(a,b), f(a,?b) -> ?b\^{}2);
        2
        b
        s(a+b, a+b -> a{*}b);
        b*a
\end{verbatim}
``Associativity'' is used to group $a+b+c$ to $(a+b)+c$:
\begin{verbatim}
        s(a+b+c, a+b -> a*b);
        b*a + c
\end{verbatim}
The next three examples use a rule set that defines the factorial function.
Substitute once:
\begin{verbatim}
        s(nfac(3), {nfac(0) -> 1, nfac(?x) -> ?x*nfac(?x-1)});
        3*nfac(2)
\end{verbatim}
Substitute twice:
\begin{verbatim}
        s(nfac(3), {nfac(0) -> 1, nfac(?x) -> ?x*nfac(?x-1)}, 2);
        6*nfac(1)
\end{verbatim}
Substitute until expression stops changing:
\begin{verbatim}
        si(nfac(3), {nfac(0) -> 1, nfac(?x) -> ?x{*}nfac(?x-1)});
        6
\end{verbatim}
Only substitute at the top level:
\begin{verbatim}
        s(a+b+f(a+b), a+b -> a*b, inf, 0);
        f(b+a) + b*a
\end{verbatim}



\subsection{temp :- exp and temp ::- exp}

If during simplification of an expression, \texttt{temp} matches some
sub-expression, then that sub-expression is replaced by \texttt{exp}.  If there
is a choice of templates to apply, the least general is used.

If an old rule exists with the same template, then the old rule is replaced by
the new rule.  If \texttt{exp} is \texttt{nil} the rule is retracted.

\texttt{temp ::- exp} is the same as \texttt{temp :- exp}, but the l.h.s. is not
simplified until the replacement is made.

\subsubsection*{Examples:}

Define the factorial function of a natural number as a recursive function and a
termination condition.  For all other values write it as a gamma function.  Note
that the order of definition is not important, as the rules are re-ordered so
that the most specific rule is tried first.  Note the use of \texttt{::-}
instead of \texttt{:-} to stop simplification of the l.h.s.  \texttt{hold} stops
its arguments from being simplified. 
\begin{verbatim}
        fac(?x \_= Natp(?x)) ::- ?x*fac(?x-1);
        hold(fac(?X-1)*?X)
        fac(0) :- 1;
        1
        fac(?x) :- Gamma(?x+1);
        gamma(?X + 1)
        fac(3);
        6
        fac(3/2);
        gamma(5/2)
\end{verbatim}



\subsection{Arep(\{rep1,rep2,\ldots\})}

In future simplifications automatically apply replacements \texttt{rep1, rep2,
  ...} until the rules are retracted.  In effect, this replaces the operator
\texttt{->} by \texttt{:-} in the set of replacements \texttt{\{rep1, rep2,
  ...\}}.


\subsection{Drep(\{rep1,rep2,..\})}

Delete the rules \texttt{rep1, rep2, ...}.

As we said earlier, the matcher has been constructed along the lines of the
pattern matcher described in McIsaac with the addition of such-that conditions
and ``semantic matching'' as described in Grief.  To make a template efficient,
some consideration should be given to the structure of the template and the
position of such-that statements.  In general the template should be constructed
so that failure to match is recognized as early as possible.  The multi-generic
symbol should be used whenever appropriate, particularly with symmetric
functions.  For further details see McIsaac.

\subsubsection*{Examples:}

\texttt{f(?a,?a,?b)} is better than \texttt{f(?a,?b,?c\_=(?a=?b))}. 
\texttt{?a+??b} is better than \texttt{?a+?b+?c...}.

The template \texttt{f(?a+?b,?a,?b)}, matched against \texttt{f(3,2,1)} is
matched as \texttt{f(?e\_=(?e=?a+?b),?a,?b)} when semantic matching is allowed.


\subsection{Switches}

\begin{description}
  \item[\texttt{TRPM}] Produces a trace of the rules applied during a
    substitution.  This is useful to see how the pattern matcher works, or to
    understand an unexpected result.
\end{description}
In general usage the following switches need not be considered:
\begin{description}
  \item[\texttt{SEMANTIC}] Allow semantic matches, e.g. \texttt{f(?a+?b,?a,?b)}
    will match \texttt{f(3,2,1)}, even though the matcher works from left to
    right.
  \item[\texttt{SYM!-ASSOC}] Limits the search space of symmetric associative
    functions when the template contains multi-generic symbols so that generic
    symbols will not function. For example \texttt{(a+b+c,?a+??b)} will return
      \begin{verbatim}
        {?a -> a, ??b -> [b,c]} or
        {?a -> b, ??b -> [a,c]} or
        {?a -> c, ??b -> [a,b]}
      \end{verbatim}
    but not \texttt{\{?a -> a+b, ??b -> c\}}, etc.  No sane template should
    require these types of matches.  However they can be made available by
    turning the switch off.
\end{description}

