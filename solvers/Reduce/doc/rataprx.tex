
\subsection{Periodic Decimal Representation}

\index{Periodic decimal representation}
The division of one integer by another often results in
a period in the decimal part. The \texttt{rational2periodic}
function in this package can recognise and represent
such an answer in a periodic representation. The inverse
function, \texttt{periodic2rational}, can also convert a
periodic representation back to a rational number.

\hypertarget{operator:RATIONAL2PERIODIC}{}
\hypertarget{operator:PERIODIC2RATIONAL}{}
\hypertarget{operator:RATIONAL}{}

\ttindextype{RATIONAL2PERIODIC}{operator}
\ttindextype{PERIODIC2RATIONAL}{operator}
\ttindextype{PERIODIC}{operator}
\begin{tabbing}
\textbf{\underline{Periodic Representation of a Rational Number}}\\[\baselineskip]


\textbf{SYNTAX:} \hspace{3mm} 
     \= \texttt{rational2periodic(n);}\\[\baselineskip]

\textbf{INPUT:}  
     \> \texttt{n} \hspace{3mm} is a rational number\\[\baselineskip]

\textbf{RESULT:}
     \> \texttt{periodic(\{a,b\} , \{c1,...,cn\})} \\[\baselineskip]
     \> where  \texttt{a/b} is the non-periodic part\\
     \> and \texttt{c1,...,cn} are the digits of the periodic part.\\[\baselineskip]


\textbf{EXAMPLE:}
    \> $59/70$ written as $0.8\overline{428571}$\\
    \>  \texttt{1: rational2periodic(59/70);}\\[\baselineskip]
    \> \texttt{periodic(\{8,10\},\{4,2,8,5,7,1\})}\\[\baselineskip]


\textbf{\underline{Rational Number of a Periodic Representation}}\\[\baselineskip]


\textbf{SYNTAX:}
     \> \texttt{periodic2rational(periodic(\{a,b\},\{c1,...,cn\}))}\\
     \> \texttt{periodic2rational(\{a,b\},\{c1,...,cn\})}\\[\baselineskip]


\textbf{INPUT:}
     \> \hspace{15mm} \texttt{a} \hspace{3mm}\= is an integer\\
     \> \hspace{15mm} \texttt{b}             \> is $1$, $-1$ or an 
                                             integer multiple of $10$\\
     \> \texttt{c1,...,cn}               \> is a list of positive digits\\[\baselineskip]

\textbf{RESULT:}
     \> A rational number.\\[\baselineskip]

\textbf{EXAMPLE:}
    \> $0.8\overline{428571}$ written as $59/70$ \\
    \> \texttt{2: periodic2rational(periodic(\{8,10\},\{4,2,8,5,7,1\}));}
\\[\baselineskip]
    \> \hspace{1mm} \texttt{59}\\
    \> \texttt{----}\\
    \> \hspace{1mm} \texttt{70}\\[\baselineskip]
    \> \texttt{3: periodic2rational(\{8,10\},\{4,2,8,5,7,1\});}
\\[\baselineskip]
    \> \hspace{1mm} \texttt{59}\\
    \> \texttt{----}\\
    \> \hspace{1mm} \texttt{70}
\end{tabbing}

Note that if \texttt{a} is zero, \texttt{b} will indicate how many places
after the decimal point that the period occurs. Note also that if the answer
is negative then this will be indicated by the sign of \texttt{a} (unless
\texttt{a} is zero in which case it is indicated by the sign of \texttt{b}).
\\[\baselineskip]
%
%
\textbf{ERROR MESSAGE}\\
%
\texttt{***** operator to be used in off rounded mode}\\
%
The periodicity of a function can only be recognised in
the \texttt{off rounded} mode. This is also true for the inverse
procedure.\\[\baselineskip]
%
%
{\large\bf EXAMPLES}
\begin{verbatim}
4: rational2periodic(1/3);

periodic({0,1},{3})

5: periodic2rational(ws);

 1
---
 3

6: periodic2rational({0,1},{3});

 1
---
 3

7: rational2periodic(-1/6);

periodic({-1,10},{6})

8: periodic2rational(ws);

  - 1
------
  6

9: rational2periodic(6/17);

periodic({0,1},{3,5,2,9,4,1,1,7,6,4,7,0,5,8,8,2})

10: periodic2rational(ws);

 6
----
 17

11: rational2periodic(352673/3124);

periodic({11289,100},{1,4,8,5,2,7,5,2,8,8,0,9,2,1,8,9,5,0,0,6,
                      4,0,2,0,4,8,6,5,5,5,6,9,7,8,2,3,3,0,3,4,
                      5,7,1,0,6,2,7,4,0,0,7,6,8,2,4,5,8,3,8,6,
                      6,8,3,7,3,8,7,9,6,4})

12: periodic2rational(ws);

 352673
--------
  3124

\end{verbatim}
%\newpage
\subsection{Continued Fractions}

\index{Continued fractions}
A continued fraction (see ~\cite{PA} \S 4.2) has the general form
{\Large
\[b_0 + \frac{a_1}{b_1 +
         \frac{a_2}{b_2+
          \frac{a_3}{b_3 + \ldots
        }}}
\;.\]
}

A more compact way of writing this is as
\[b_0 + \frac{a_1|}{|b_1} + \frac{a_2|}{|b_2} + \frac{a_3|}{|b_3} + \ldots\,.\]
\\
%
\ttindextype{CONTFRAC}{operator}
This is represented in {\small REDUCE} as
\[{\tt
   contfrac(\text{\sl Rational\hspace{2mm} approximant},
                \{b0, \{a1,b1\}, \{a2,b2\},.....\})
}\]

\hypertarget{CONTFRAC:operator}{}
\hypertarget{CFRAC:operator}{}
\ttindextype{CFRAC}{operator}
\begin{tabbing}
\\
\textbf{SYNTAX:} \hspace{5mm} 
\= \texttt{cfrac(number);}\\
\> \texttt{cfrac(number,length);}\\
\> \texttt{cfrac(f, var);}\\
\> \texttt{cfrac(f, var, length);}\\[\baselineskip]

\textbf{INPUT:}
\> \texttt{number} \hspace{3mm} \= is any real number\\
\> \texttt{f}                   \> is a function\\
\> \texttt{var}                 \> is the function variable\\
%\> \texttt{length}              \> is the upper bound of the number\\ 
%\>                           \> of \{ai,bi\} returned (optional)\\[\baselineskip]
\end{tabbing}

\textbf{Optional Argument: \texttt{length}}\\
%
The \texttt{length} argument is optional. 
For an NON-RATIONAL function input the \texttt{length} argument specifies
the number of ordered pairs, $\{a_i,b_i\}$, to be 
returned. It's default value is five.
For a RATIONAL function input the
\texttt{length} argument can only truncate the answer, it cannot
return additional pairs even if the precision is increased.
The default value is the complete continued fraction of the
rational input. For a NUMBER input the default value is 
dependent on the precision of the session, and the
\texttt{length} argument will only take effect if it has a smaller
value than that of the number of ordered pairs which the default
value would return.\\[\baselineskip]
%
%
%
%\newpage
\large{\textbf{EXAMPLES}}
\begin{verbatim}
13: cfrac(23.696);

          2962
contfrac(------,{23,{1,1},{1,2},{1,3},{1,2},{1,5}})
          125


14: cfrac(23.696,3);

          237
contfrac(-----,{23,{1,1},{1,2},{1,3}})
          10

15: cfrac pi;


          1146408
contfrac(---------,
          364913

         {3,{1,7},{1,15},{1,1},{1,292},{1,1},{1,1},{1,1},{1,2},{1,1}})

16: cfrac(pi,3);

          355
contfrac(-----,{3,{1,7},{1,15},{1,1}})
          113

17: cfrac(pi*e*sqrt(2),4);

          10978
contfrac(-------,{12,{1,12},{1,1},{1,68},{1,1}})
           909

18: cfrac((x+2/3)^2/(6*x-5),x,1);

             2
          9*x  + 12*x + 4    6*x + 13      24*x - 20
contfrac(-----------------,{----------,{1,-----------}})
             54*x - 45          36             9

19: cfrac((x+2/3)^2/(6*x-5),x,10);

             2
          9*x  + 12*x + 4    6*x + 13      24*x - 20
contfrac(-----------------,{----------,{1,-----------}})
             54*x - 45          36             9


20: cfrac(e^x,x);



           3      2
          x  + 9*x  + 36*x + 60
contfrac(-----------------------,{1,{x,1},{ - x,2},{x,3},{ - x,2},{x,5}})
               2
            3*x  - 24*x + 60

21: cfrac(x^2/(x-1)*e^x,x);

           6      4    2
          x  + 3*x  + x
contfrac(----------------,{0,
             4    2
          3*x  - x  - 1

                 2            2       2       2       2
            { - x ,1}, { - 2*x ,1}, {x ,1}, {x ,1}, {x ,1}})

22: cfrac(x^2/(x-1)*e^x,x,2);

             2
             x              2           2
contfrac(----------,{0,{ - x ,1},{ - 2*x ,1}})
             2
          2*x  - 1

\end{verbatim}

%\newpage
\subsection{Pad\'{e} Approximation}
\index{Pad\'{e} Approximation}

The Pad\'{e} approximant represents a function by the ratio of two 
polynomials. The coefficients of the powers occuring in the polynomials 
are determined by the coefficients in the Taylor series
expansion of the function (see ~\cite{PA}). Given a power series
\[ f(x) = c_0 + c_1 (x-h) + c_2 (x-h)^2 \ldots \]
and the degree of numerator, $n$, and of the denominator, $d$,
the \texttt{pade} function finds the unique coefficients 
$a_i,\, b_i$ in the Pad\'{e} approximant 
\[ \frac{a_0+a_1 x+ \cdots + a_n x^n}{b_0+b_1 x+ \cdots + b_d x^d} \; .\]
\\[\baselineskip]
%
\hypertarget{PADE:operator}{}
\ttindextype{PADE}{operator}
\begin{tabbing}
\textbf{SYNTAX:} \hspace{5mm}\= \texttt{pade(f, x, h, n, d);}\\[\baselineskip]

\textbf{INPUT:}
\> \texttt{f} \hspace{3mm} \= is the funtion to be approximated\\
\> \texttt{x}             \> is the function variable\\
\> \texttt{h}             \> is the point at which the approximation is\\ 
\>                     \> evaluated\\
\> \texttt{n}             \> is the (specified) degree of the numerator\\
\> \texttt{d}             \> is the (specified) degree of the denominator\\[\baselineskip]


\textbf{RESULT:} 
\> Pad\a'{e} Approximant, ie. a rational function.\\[\baselineskip]
\end{tabbing}


\textbf{ERROR MESSAGES}\\
%
\texttt{***** not yet implemented}\\
%
The Taylor series expansion for the function, f, has not yet
been implemented in the {\small REDUCE} Taylor Package.\\[\baselineskip]
%
%
\texttt{***** no Pade Approximation exists}\\
%
A Pad\'{e} Approximant of this function does not exist.\\[\baselineskip]
%
%\newpage
\texttt{***** Pade Approximation of this order does not exist}\\
%
A Pad\'{e} Approximant of this order (ie. the specified
numerator and denominator orders) does not exist but one
of a different order may exist.\\[\baselineskip]
%
%
\large{\textbf{EXAMPLES}}

\begin{verbatim}

23: pade(sin(x),x,0,3,3);

          2
 x*( - 7*x  + 60)
------------------
       2
   3*(x  + 20)

24: pade(tanh(x),x,0,5,5);

     4        2
 x*(x  + 105*x  + 945)
-----------------------
      4       2
 15*(x  + 28*x  + 63)

25: pade(atan(x),x,0,5,5);

        4        2
 x*(64*x  + 735*x  + 945)
--------------------------
         4       2
 15*(15*x  + 70*x  + 63)

26: pade(exp(1/x),x,0,5,5);

***** no Pade Approximation exists

27: pade(factorial(x),x,1,3,3);

***** not yet implemented

28: pade(asech(x),x,0,3,3);

            2                        2                 2
- 3*log(x)*x  + 8*log(x) + 3*log(2)*x  - 8*log(2) + 2*x
--------------------------------------------------------
                          2
                       3*x  - 8

29: taylor(ws-asech(x),x,0,10);

               11  
log(x)*(0 + O(x  ))

     13    6     43    8    1611    10      11
 + (-----*x  + ------*x  + -------*x   + O(x  ))
     768        2048        81920

30: pade(sin(x)/x^2,x,0,10,0);

***** Pade Approximation of this order does not exist

31:  pade(sin(x)/x^2,x,0,10,2);

     10        8         6           4            2
( - x   + 110*x  - 7920*x  + 332640*x  - 6652800*x

  + 39916800)/(39916800*x)

32: pade(exp(x),x,0,10,10);



  10        9         8           7            6
(x   + 110*x  + 5940*x  + 205920*x  + 5045040*x

              5               4                3
  + 90810720*x  + 1210809600*x  + 11762150400*x

                 2
  + 79394515200*x  + 335221286400*x + 670442572800)/

      10        9         8           7            6
    (x   - 110*x  + 5940*x  - 205920*x  + 5045040*x

                     5               4
         - 90810720*x  + 1210809600*x 

                        3               2
         - 11762150400*x + 79394515200*x  

         - 335221286400*x + 670442572800)

33: pade(sin(sqrt(x)),x,0,3,3);
        
(sqrt(x)*
            3            2
  (56447*x  - 4851504*x  + 132113520*x - 885487680))\

              3         2
    (7*(179*x  - 7200*x  - 2209680*x - 126498240))
\end{verbatim}


\begin{thebibliography}{9}
\bibitem{PA} Baker(Jr.), George A. and Graves-Morris, Peter:\\
{\it Pad\'{e} Approximants, Part I: Basic Theory},
(Encyclopedia of mathematics and its applications, Vol 13,
Section: Mathematics of physics),
Addison-Wesley Publishing Company, Reading, Massachusetts, 1981.
\end{thebibliography}
