\appendix

\chapter{CDs and Symbols handled by the Interface}

\begin{tabular}{p{2in} p{2in}}

{\bf alg1} & {\tt one, zero}\\ \\

{\bf arith1 } & {\tt abs, conjugate, divide, minus, plus, power, product, root, sum, times, unary\_minus}\\ \\

{\bf arith2 } & {\tt arg, inverse, times}\\ \\


{\bf calculus1} & {\tt defint, diff, int, partialdiff}\\ \\


{\bf fns1 }& {\tt inverse, lambda}\\ \\


{\bf integer1 }& {\tt factorial, gcd , quotient, rem}\\ \\


{\bf interval1 }& {\tt integer\_interval, interval, interval\_cc, interval\_co, interval\_oc, interval\_oo}\\ \\

{\bf limit1 }& {\tt both\_sides, above, below, limit, null}\\ \\

{\bf linalg1 }& {\tt matrix, outerproduct, scalarproduct, vector, vectorproduct}\\ \\

\end{tabular}


\begin{tabular}{p{2in} p{2in}}


{\bf linalg2 }& {\tt vector}\\ \\


{\bf linalg3 }& {\tt determinant, matrix\_selector, selector, size, transpose, vector\_selector}\\ \\

{\bf list1 }& {\tt list}\\ \\


{\bf logic1 }& {\tt and, false, implies, not, or, true, xor}\\ \\


{\bf logic2 }& {\tt equivalent}\\ \\


{\bf minmax1 }& {\tt max, min}\\ \\


{\bf multiset1 }& {\tt in, intersect, multiset, notin, notprsubset, notsubset, prsubset, set, setdiff, subset, union}\\ \\

{\bf nums1 }& {\tt based\_integer, complex\_cartesian, complex\_polar, e, gamma, i, imaginary, infinity, nan, pi, rational,
real}\\ \\


{\bf omtypes }& {\tt float, integer}\\ \\


{\bf quant1 }& {\tt exists, forall}\\ \\


{\bf relation1 }& {\tt eq, geq, gt, leq, lt, neq}\\ \\


{\bf relation2 }& {\tt approx}\\ \\


{\bf set1 }& {\tt in, intersect, notin, notprsubset, notsubset, prsubset, set, setdiff, subset, union}\\ \\


{\bf stats1 }& {\tt mean, median, mode, moment, sdev, variance}\\ \\

\end{tabular}


\begin{tabular}{p{2in} p{2in}}

{\bf transc1 }& {\tt arccos, arccosh, arccot, arccoth, arccsc, arccsch, arcsec, arcsech, arcsin, arcsinh, arctan, arctanh, cos,
cosh, cot, coth, csc, csch, exp, ln, log, sec, sech, sin, sinh, tan, tanh}\\ \\


{\bf transc2 }& {\tt arccot, arccoth, arccsc, arccsch, arcsec, arcsech, arcsinh, arctanh}\\ \\


{\bf typmml }& {\tt complex\_cartesian\_type, complex\_polar\_type, constant\_type, fn\_type, integer\_type, list\_type,
matrix\_type, rational\_type, real\_type, set\_type, type, vector\_type}\\ \\


{\bf veccalc1 }& {\tt curl, divergence, grad, laplacian}\\ \\


\end{tabular}

