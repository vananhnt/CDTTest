\documentclass[12pt,a4paper]{article}

\usepackage[centertags,reqno]{amsmath}
\usepackage{verbatim}

\newcommand{\OCCAL}{{\sf\bf OCCAL}}
\newcommand{\REDUCE}{{\sf\bf REDUCE}}
\newcommand{\R}{\mbox{{\sf\bf R}}}
\newcommand{\FORTRAN}{{\sf FORTRAN}}
\newcommand{\RATFOR}{{\sf RATFOR}}
\newcommand{\Pascal}{{\sc Pascal}}
\newcommand{\C}{{\sf C}}
\newcommand{\GENTRAN}{{\sf GENTRAN}}
\newcommand{\MULCON}{{\sf MULCON}}
\newcommand{\SCOPE}{{\sf SCOPE}}
\newcommand{\perl}{{\sf perl}}

\renewcommand{\and}{\quad}

\newlength{\dummy}
\settowidth{\dummy}{Appendix~\mbox{}}
\addtolength{\dummy}{3mm}

\title{OCCAL\\
       A mixed symbolic-numeric\\
       Optimal Control CALculator}

\date{ }

\author{Rainer Sch\"opf
        \and
        Peter Deuflhard
        \\
        Konrad-Zuse-Zentrum\\
        f\"ur Informationstechnik Berlin\\
        Heilbronner Str. 10\\
        W-1000 Berlin 31\\
        Federal Republic of Germany}

\begin{document}

\maketitle

\begin{abstract}

  The numerical solution of optimal control problems by indirect
  methods (such as multiple shooting or collocation) requires a
  considerable amount of analytic calculation to establish a
  numerically tractable system.  These analytic calculations, though
  being rather tedious in realistic examples, are nowadays mostly
  still done by hand---and thus prone to calculation errors.  The
  paper aims at automating this analytic processing to a reasonable
  extent by means of a modern symbolic manipulation language (here:
  \REDUCE).  In its present stage of development the package \OCCAL{}
  (mnemotechnically for \underline{O}ptimal \underline{C}ontrol
  \underline{CAL}culator) permits an interactive use, covering tasks
  like automatic determination of control and, in case of a singular
  control, of its order.
%
  In simpler problems, the present version of \OCCAL{} automatically
  produces the full subroutine input for a MULtiple shooting code
  (\MULCON) with adaptive numerical CONtinuation.\\
%
  In more complicated problems where singular sub-arcs may occur or
  where the sequence of sub-arcs of the optimal trajectory is unclear
  \OCCAL{} is a significant help in reducing analytic pre-processing.
  Examples illustrate the performance of \OCCAL/\MULCON.

%  We describe \OCCAL, a integrated symbolic-numeric calculus system
%  for the treatment of optimal control problems. The system is not
%  only able to do a lot of simple but tedious analytic calculations,
%  but uses also advanced techniques to calculate as much as possible
%  automatically. Optimized code for the numeric boundary value problem
%  solver is generated as well.
\end{abstract}
\vspace*{0.5cm}
Preprint SC 91-13
\newpage

\tableofcontents

\newpage

\section{Introduction}

Optimal Control problems appear in a wide variety of important
scientific and technical applications.  In spite of the rather high
standards of {\em numerical\/} algorithms for such problems
\cite{Bock1,Bock2,Bock3,bulirsch,DFK2} tackling such a problem is
still a complicated task.  The reason for that is that certain steps in
the analytical preparation of the calculation, which are simple in
principle, may be very elaborate and can lead to rather complex
expressions.  Implementation of these into numerical code by hand is
tiresome and error-prone.

The purpose of our system \OCCAL{} to be presented here is to overcome
this tedious preparation phase. We intend to solve as many of the
arising problems as possible.

%To this end we employ a combined symbolic-numeric system, where the
%first step is to symbolically analyze the given problem, using the
To this end we employ a combined symbolic system that
symbolically analyzes the given problem, using the
Computer Algebra System \REDUCE.  In passing, this approach
additionally permits to generate {\em optimized\/} numerical
subprograms for use with existing numerical solvers for boundary value
problems. These are employed in the second step to obtain numerical
solutions.

Symbolic computation (as most computations) is often applied to tasks
that are simple in principle, but tedious to carry out by hand.
Consequently, often only some well defined steps in a calculation are
done with the help of a Computer Algebra System. The results are then
treated more or less by hand.

%\OCCAL{}, however is an integrated system in which the symbolic and
%numeric processors interact with each other, thereby allowing
%interchange of information between each other and with the user. The
%advantage of this approach is obvious: at each point in a calculation,
%the optimal method to solve the task at hand can be used. Examples of
%this are differentiation of functions, which can be ill-behaved if
%done numerically, or matrix computations, which can usually be done
%much faster by a numerical processor. Additionally, the automatic
%exchange of information reduces the number of errors the user may
%introduce into the calculation.

It should be stressed at this point that the symbolic processor in
\OCCAL{} is not only used for the tasks mentioned as ``tedious, but
simple in principle''.  Nowadays, advanced symbolic techniques are
available, e.g., for the solution of systems of nonlinear equations or
for computations with inequalities. This can be used, for example, to
determine the sign of a function of several variables over a given
range.

The need for these advanced methods within \OCCAL{} has not only led
to their improvement and adaptation to the problems at hand, but also
to their inclusion into the recently released new version of the
Computer Algebra System \REDUCE. Therefore these techniques are now
available to a much greater community than before, and hopefully also
for a much wider area of applications.

In Section \ref{sec:intro} below the typical structure of optimal
control problems is presented\allowbreak---mainly to fix notations and to
elucidate the kind of analytical work involved in the pre-numerical
stage. On this basis those tasks that can be automated are listed in
detail in Section \ref{sec:operation}---together with the
corresponding details of symbolic devices. Finally in Section
\ref{sec:examples} the performance of \OCCAL{} in a few well-known
examples is documented.

Of course, the full use of such a software system can only be
estimated by actually trying it!


\section{A Short Introduction \protect\\ to Optimal Control Problems}

\label{sec:intro}

Consider a system that is described by a number of time-dependent
state variables $y$. Let $y\in K \subset C^{1}$, where $K$ is a
suitable class of continously differentiable functions, and $y\colon
[a,b]\rightarrow \R^{n}$.  This system is controlled by certain {\em
control variables\/} $u\in C^{0}$ piecewise, $u\colon [a,b]\rightarrow
\R^{k}$, and governed by the following system of differential
equations
\begin{equation}
  y' = f(t,y,u), \label{dyn-sys}
\end{equation}
the with separated boundary conditions $y(a)=y_{a}$ and
\begin{equation}
  r(b,y(b)) =0, \quad r\colon \R^{n+1}\rightarrow \R^{p},
   \quad  p\leq n.
\end{equation}
The task is to minimize the integral
\begin{equation}
  I[u] := \int\limits_{a}^{b} f_{0}(y,u,t) dt
\end{equation}
over $K$.
%We couple the differential equations and the boundary conditions
%to the functional using Lagrange multipliers $\lambda$:
%\begin{equation}
%  \hat I[u,y,y',\lambda] := \Phi(b,y(b)) + \nu^{T} r
%      + \int_{a}^{b} \left(\sum_{i=1}^{n} \lambda_{i} (y'_{i} -
%      f_{i}(t,y,u)) \right) dt
%      \label{full-func}
%\end{equation}
Similar to the calculus of variations there is the so-called {\em
Hamiltonian\/}
\begin{equation}
  H(t,y,\lambda,u) := \sum_{i=1}^{n} \lambda_{i} f_{i}(t,y,u)
     +\lambda_{0} f_{0}(t,y,u) \label{Hamiltonian},
\end{equation}
where $\lambda(t) = (\lambda_{1}(t),\ldots,\lambda_{n}(t))$
denotes the {\em adjoint variables}.

The minimum principle of Pontrjagin states then that a necessary condition for
$(y_{0}, u_{0})$ being a solution of the optimal control problem is
\begin{equation}
  H(t,y_{0},\lambda,u_{0}) = \min_{v} H(t,y_{0},\lambda,v) \label{pontrjagin}
\end{equation}
with $v$ fulfilling the conditions.  Calculating the first variation
leads to the so-called {\em canonical equations\/}
\begin{eqnarray}
  y'_{i} &=& H_{\lambda_{i}} = f_{i}(t,y,u), \quad i = 1,\ldots,n, \nonumber\\
  \lambda'_{i} &=& - H_{y_{i}} = - \frac{\partial f_{0}}{\partial y_{i}}(t,y,u)
        - \sum_{j=1}^{n} \lambda_{i}
          \frac{\partial f_{j}}{\partial y_{i}}(t,y,u), \quad i =
          1,\ldots,n.           \label{can-eqs}
\end{eqnarray}
We number the control variables $u_{i}$ in such a way that the first
$m$ components (denoted by $u^{(n)}$) appear nonlinearly in $f$, while
the others (denoted by $u^{(l)}$) appear only linearly, so that $f$
splits in the form
\begin{eqnarray}
  f(t,y,u) &=:& g(t,y,u^{(n)}) + h(t,y,u^{(n)}) u^{(l)}\:, \nonumber \\
    && u^{(n)} \in \R^{m}, u^{(l)} \in \R^{q}, \quad m+q=k \label{u-dist}\:.
\end{eqnarray}
This splitting carries over to the Hamiltonian $H$ as well.  Since
$u^{(l)}$ appears only linearly in $f$ and $H$ (cf.\ (\ref{u-dist}))
minimization of $H$ can be done in two steps:
\begin{enumerate}
  \item Determination of $u^{(n)}$ from
  \begin{eqnarray}
    && H_{u^{(n)}} = 0 \nonumber \\
    && H_{u^{(n)}u^{(n)}} \mbox{ positive (semi-)definite} \label{sys-nonlin}
  \end{eqnarray}
  (Here $H_{u^{(n)}}$ denotes the Fr\'echet derivatives.)  One obtains
  an analytic expression $u^{(n)} = u^{(n)}(t,y,\lambda)$ which can be
  substituted into the original expressions (\ref{can-eqs}). Thus
  $u^{(n)}$ essentially drops out of the set of variables.
  \item Determination of $u^{(l)}$:\\
        After substitution of $u^{(n)}$ the Hamiltonian can be written as
        \begin{equation}
          H(t,y,\lambda,u^{(n)}(t,y,\lambda),u^{(l)})
            =: H_{0}(t,y,\lambda)
            +  \sum_{i=1}^{q} S_{i}(t,y,\lambda) u_{m+i}\:.
        \end{equation}
        Obviously we have
        \begin{equation}
          S_{i}(t,y,\lambda) = H_{u_{m+i}}(t,y,\lambda,u^{(n)}(t,y,\lambda),u^{(l)})
        \end{equation}
        with $i=1,\ldots,q$. 
        The condition (\ref{pontrjagin}) is then equivalent to
        \begin{equation}
          S_{i}(t,y_{0},\lambda) \cdot u_{0}(t) = \min_{v} S_{i}(t,y_{0},\lambda) \cdot v(t),
        \end{equation}
      which is a linear optimization problem. 
      It is therefore necessary that the control variables $u^{(l)}$ are subject to
      control constraints usually of the form
        \begin{eqnarray}
          && \alpha_{i} \leq u_{i} \leq \beta_{i}, \quad i=m+1,\dots,k
          \nonumber \\
          && \alpha_{i}, \beta_{i} \in \R.
        \end{eqnarray}
      The further procedure
      depends on which one of the following two cases holds:
        \begin{enumerate}
          \item $S_{i} \not\equiv 0$, i.e.\ $S_{i}$ has only isolated zeros.
               This is the case of a {\em bang-bang control}.
               From the minimum principle we deduce
                 \begin{equation}
                   u_{m+i} = \left\{\begin{array}{l@{\quad}l}
                                  \alpha_{m+i} & \mbox{for $S_{i} > 0$} \\
                                  \beta_{m+i} & \mbox{for $S_{i} < 0$}
                                \end{array}\right.
                 \end{equation}
               The $S_{i}$ are called {\em switching functions}. 
               A change of sign can only occur in the {\em switching points\/}
               $\tau^{i}$:
                 \begin{equation}
                   S_{i}(\tau^{i},y(\tau^{i}),\lambda(\tau^{i})) = 0.
                 \end{equation}
          \item $S_{i} \equiv 0$ within a non-empty interval. This is the so-called {\em singular\/} case.
                Let us assume, for simplicity, that $q=1$, i.e.\ there
                is only one linear control $u$, and that there is only
                one such interval.  Define the following functions
                recursively by
                 \begin{eqnarray}
                   S^{(0)}(t,y,\lambda) &:=& S(t,y,\lambda) \\
                   S^{(k+1)}(t,y,\lambda) &:=& \frac{d
                   S^{(k)}(t,y,\lambda)}{dt} \nonumber \\
                        &=& S^{(k)}_{y} \cdot \dot{y} +
                        S^{(k)}_{\lambda} \cdot \dot{\lambda} + S^{(k)}_{t}\\
                        &=&  S^{(k)}_{y} \cdot H_{\lambda} -
                             S^{(k)}_{\lambda} \cdot H_{y}  +  S^{(k)}_{t} \nonumber
                 \end{eqnarray}
                Define $m$ to be the smallest index such that
                $\partial S^{(m)}/\partial u \not\equiv 0$. Then
                theory \cite{th1,th2,th3} shows that $m = 2r$ is an even
                number. $r$ is called {\em order\/} of the singular
                control.
                
                Since $u^{(l)}$ appears only linearly in $H_{y}$ and $H_{\lambda}$ it follows
                that $S^{(m)}$ is of the form
                 \begin{equation}
                   S^{(m)}(t,y,\lambda) = A(t,y,\lambda) + B(t,y,\lambda)\cdot u(t).
                 \end{equation}
                For $B(t,y,\lambda)\not=0$ the singular control can be
                determined to be
                 \begin{equation}
                   u_{0}(t,y,\lambda) = - \frac{A(t,y,\lambda)}{B(t,y,\lambda)} \label{singular-control}
                 \end{equation}
                In certain cases $u_{0}$ does not depend on the adjoint
                variables $\lambda$; this is called a {\em feedback
                control}.  For an optimal trajectory $y(t)$ and an
                optimal control $u_{0}(t)$ of order $k = 2r$ holds the
                {\em generalized Legendre-Clebsch condition\/} \cite{th3}
                  \begin{equation}
                    0 \leq (-1)^{r} \cdot \frac{\partial}{\partial u}
                     \left(\frac{d^{2r}}{dt^{2r}}H_{u}(t)\right)
                     =  (-1)^{r} \cdot  B(t,y(t),\lambda(t)),
                  \end{equation}
                In the case of a feedback control a strict inequality holds.
        \end{enumerate}
\end{enumerate}
Finally we can determine the boundary condition for the $\lambda_{i}$ as
\begin{equation}
  \left. \lambda_{i} \right |_{b} = \Phi_{y_{i}(b)} + \nu^{T} r_{y_{i}(b)},
\end{equation}
where the $\nu$ are Lagrange multipliers.

\vspace*{3mm}
\noindent {\bf Numerical solution by means of Multiple Shooting.}\\
After $u^{(n)}$ has been substituted back into the canonical equations
(\ref{can-eqs}) we can proceed to a {\em numerical\/} solution of the
resulting boundary value problem.  If there is a linear control
$u^{(l)}$ we have to take the structure of the optimal control problem
into account, especially the number $m_{s}$ and position of the switching
points $\tau_{j}$ defined by the equations
\begin{equation}
  S_{i}(\tau_{j},y(\tau_{j}),\lambda(\tau_{j})) = 0,\quad
   j = 1,\ldots,m_{s}.
\end{equation}
Since the position of these points is in general not known beforehand,
we introduce them as additional parameters
\begin{displaymath}
  \tau_{1} < \tau_{2} < \ldots < \tau_{m_{s}}
\end{displaymath}
and transform every sub-interval into an interval of the same length,
i.e.,
\begin{eqnarray}
         [a,\tau_{1}] \rightarrow [0,1] \colon
                 \bar t &:=& \frac{t - a}{\tau_{1} - a} \nonumber\\ \relax
  [\tau_{1},\tau_{2}] \rightarrow [1,2] \colon
                 \bar t &:=& \frac{t - \tau_{1}}{\tau_{2} - \tau_{1}} + 1 \nonumber\\ \relax
    \vdots \\ \relax
  [\tau_{m_{s}}, b] \rightarrow [m_{s},m_{s}+1] \colon
                 \bar t &:=& \frac{t - \tau_{m_{s}}}{b - \tau_{m_{s}}} + m_{s} \nonumber\\
  y(t) \rightarrow \bar y(\bar t),&& \lambda(t) \rightarrow \bar\lambda(\bar t) \nonumber
\end{eqnarray}
This leads to extra factors in the right hand side of the differential
equations which then additionally depend on one parameter (for the
border intervals) or on two parameters (for the intermediate
intervals). The parameters $\tau_{1},\ldots,\tau_{m_{s}}$ are
determined by the {\em inner point conditions}
\begin{equation}
  S_{i}(\tau_{j},\bar y(\tau_{j}),\bar \lambda(\tau_{j})) = 0,\quad
   j = 1,\ldots,m_{s}.
\end{equation}
It may happen that the switching structure changes during the
numerical iteration. In that case the whole procedure has to be
restarted with the new structure taken into account.

Thus the problem has been reduced to a parameter dependent boundary
value problem of the general form:
\begin{eqnarray}
  y' &=& f(y;p) \\
  r(y(a),\ldots{},y(b);p) &=& 0.
\end{eqnarray}

The associated multiple shooting Jacobian then has the general block
structure
\begin{equation}
  \left[
    \begin{array}{cccccc}
      G_{1} & - I & \cdot & \cdot & P_{1} \\
      \cdot & \cdot & \cdot & \cdot & \cdot \\
      \cdot & \cdot & \cdot & \cdot & \cdot \\
      \cdot & \cdot & G_{m-1} & -I & P_{m-1} \\
      R_{1} & \cdot & R_{m-1} & R_{m} & P_{m}
    \end{array}
  \right] \label{Jacobian}
\end{equation}
in terms f the sub-matrices
\begin{eqnarray}
  G_{j} &:=& \left. W(t_{j+1},t_{j}) \right|_{y(t|x_{j},p)} \nonumber \\
  R_{j} &:=& \left. \frac{\partial r}{\partial y(t_{j})} \right|_{y(t|x_{j},p)} \nonumber \\
  P_{j} &:=& \left. P(t_{j+1},t_{j}) \right|_{y(t|x_{j},p)} \\
  P_{m} &:=& \left. \frac{\partial r}{\partial p} \right|_{y(t|x_{j},p)} \nonumber.
\end{eqnarray}
Here the matrices $W(t_{j+1},t_{j})$ are the {\em Wronskian matrices}
\begin{equation}
  W(t,t_{j}) = \frac{\partial y(t)}{\partial y(t_{j})}
\end{equation}
which are the solutions of the {\em variational equation}
\begin{equation}
  \frac{dW(t,t_{j})}{dt} = f_{y}(y(t|x_{j},p);p) W(t,t_{j}),\quad W(t_{j},t_{j}) = I,
\end{equation}
and the {\em sensitivity matrices\/} $P_{j} = P(t_{j+1},t_{j})$ are
the solutions of the generalized variational equation
\begin{equation}
  \frac{dP(t,t_{j})}{dt} = f_{y}(y(t|x_{j},p);p) P(t,t_{j}) + f_{p}(y(t|x_{j},p);p)
\end{equation}
with initial value
\begin{equation}
  P(t_{j},t_{j}) = 0.
\end{equation}


\section{Application of symbolic computation\protect\\
         to optimal control problems}

\label{sec:operation}

Summarizing the calculation steps outlined in the previous section we
have the following sub-tasks
\begin{itemize}
  \item Calculation of Hamiltonian.
  \item Identification of nonlinear versus linear control.
  \item Determination of nonlinear control by solving the system (\ref{sys-nonlin}).
  \item Substitution into differential equations (\ref{can-eqs}).
  \item Determination of linear control.
  \item In the case of bang-bang control: determination of the switching points.
  \item In the case of singular control: determination of the order of the control
        and test of the generalized Legendre-Clebsch condition.
  \item Generation of boundary conditions for the adjoint variables.
  \item Determination of the variational equations.
  \item Code generation (with optional optimization) for the differential equations,
        the boundary conditions, and the variational equations.
\end{itemize}
Some of these steps can easily be executed by {\em any\/} Computer
Algebra System (). Examples of this type of tasks are the computation of
the Hamiltonian (\ref{Hamiltonian}), of the canonical equations
(\ref{can-eqs}), of the derivatives $H_{u^{(n)}}$ and $H_{u^{(n)}u^{(n)}}$,
or the substitution of the solution $u^{(n)}$ into the Hamiltonian.

The \OCCAL{} system, however, does much more than that. We head at
accomplishing all the aforementioned tasks automatically, and have
already succeeded to a great extent.  Let us first give a short
overview of the operation of our system.

The first thing for the user to do is to formulate his problem in a
form that is suitable for processing with \OCCAL.
Since the symbolic program is written in \REDUCE, input must be given
in \REDUCE's Algol-like syntax. Even though this syntax is not too
difficult to learn and understand, it is an unnecessary burden to the
user who is not necessarily familiar with this type of program.

Therefore we decided to allow the input to be in a very natural,
user-friendly form, with automatic translation into \REDUCE{} syntax.
To this end we developed a pre-processor, using the utility \perl{}
\cite{perl}. We chose \perl{} due to its wide-spread availability and
excellent string manipulation facilities that allowed us to implement
the program in about one day's time.

The basic idea is to provide an easy-to-understand input file format,
with keywords to identify the different sections, and some freedom for
the user to choose a formulation that is most suitable to his or her
own way of thinking. Furthermore, we want to keep in mind the
possibility of a future system where this input file is only an
intermediate step in a much more sophisticated system with, e.g., a
graphical user interface. A description of the input file format is
given in appendix~\ref{java.app:perl-file}.

This pre-processing step does already some simple consistency checks:
constants, parameters and variables may be defined only once,
differential equations may only be given for already defined dynamic
variables, and so on. If there was no error, it outputs a complete
definition of the problem in \REDUCE{} syntax to a file, taking care
of arranging the statements in the correct order.  This file is
then read by the \REDUCE{} program and processed symbolically.

It should be stressed that this output is well-readable: the
experienced user may want to intervene at that point instead of
changing the input file.

In the second step, the symbolic program tackles the tasks outlined
above.  First the Hamiltonian (\ref{Hamiltonian}) is calculated. This
object is analyzed and the nonlinear and linear controls $u^{(n)}$ and
$u^{(l)}$ identified. 

The determination of the nonlinear control is done in three steps, of
which only the first one is straightforward in general:
\begin{itemize}
  \item Calculation of $H_{u^{(n)}}$.
        This involves only a series of symbolic differentiations, a
        standard problem for any Computer Algebra System.
  \item Solution of the system of equations.
        Here advanced symbolic techniques are used: algebraic
        equations are solved using the methods of Groebner bases,
        transcendental equations may be reduced to algebraic ones by a
        suitable change of variables. For example, the transcendental
        equation
        \begin{equation}
          A \sin u + B \cos u = C \label{sin-cos}
        \end{equation}
        may be converted to a system of algebraic equations by substituting
        \begin{eqnarray*}
          x = \sin u \\
          y = \cos u
        \end{eqnarray*}
      and adding the relation $x^{2}+y^{2}-1=0$.
      \typeout{******************************}
      \typeout{Einfuegen: test der direkten steuerung delta.}
      \typeout{******************************}
  \item Determination of the sign of $H_{u^{(n)}u^{(n)}}$.
        This is done using new symbolic techniques developed at ZIB.
        Two different methods are available: the simpler one of these
        tries to deduce the sign of the expression at hand by breaking
        it down into its parts and determining the sign of those.
        Though this seems to be a rather simple algorithm, it is
        nevertheless very powerful and can be applied to a wide range
        of expressions.  If this is not sufficient, the user will be
        queried for the missing information.
%
%        If this is not sufficient, a more complicated method can be
%        used: given a function of several variables, and the domains of
%        these variables, it determines the range of the function.
\end{itemize}
Finally, the solution is substituted into the Hamiltonian and the
differential equations. This uses again very much the standard
facilities of the \REDUCE{} system for differentiation, substitution
and extraction of parts of expressions.

If there is no linear control $u^{(l)}$, all that remains to do is to
solve the system of differential equations. To this end, \OCCAL{}
generates code for use with a numerical code for solving boundary
value problems. The code generation step uses the \GENTRAN{}
translator for its work; this package has interfaces for \FORTRAN,
\RATFOR, \Pascal, and \C{} programming languages.  

In our examples we use exclusively \C; our numerical BVP solver is
\MULCON{} \cite{mulcon} which was translated to \C{}
semi-automatically, first using a FORTRAN-to-C translator and then
beautifying the automatically produced code by hand to make it better
readable.

Several functions are generated in the target language:
\begin{itemize}
  \item Calculation of boundary values $r$.
  \item Calculation of differential equations $f$.
  \item Calculation of $f_{y}$ to solve the variational equation
        (for Wronskian approximations $G_{j}$, $P_{j}$).
  \item Calculation of the matrices of boundary condition derivatives
        $R_{1} = r_{y(a)}$ and $R_{m} = r_{y(b)}$.
\end{itemize}
These are exactly the functions that are needed to compute the
Jacobian (\ref{Jacobian}).

\begin{figure}[bt]
  \setlength{\unitlength}{0.5mm}
  \newcommand{\smiley}{%
    \fbox{\setlength{\unitlength}{0.5pt}%
          \begin{picture}(30,40)
            \put(8,30){\circle*{2}}
            \put(22,30){\circle*{2}}
            \put(15,10){\line(0,1){20}}
            \put(1,5){\line(1,0){20}}
          \end{picture}}}

  \begin{center}
  \begin{picture}(120,170)(0,0)
    \put(10,140){\framebox(40,25){User input}}
    \put(30,110){\oval(45,15)}
    \put(10,110){\makebox(40,0){\small Pre-processor}}
    \put(30,070){\oval(45,15)}
    \put(10,070){\makebox(40,0){\small Symbolic}}
    \put(10,010){\framebox(40,25){\small Numeric}}
    \put(80,100){\dashbox(40,25){\Large User\ \smiley}}
   \thicklines
    \put(97,127){\vector(-3,2){44}}
    \put(97,097){\vector(-3,-2){44}}
    \put(53,070){\vector(+3,+2){44}}
    \put(29,138){\vector(0,-1){19}}
    \put(31,138){\vector(0,-1){19}}
    \put(29,100){\vector(0,-1){20}}
    \put(31,100){\vector(0,-1){20}}
    \put(29,060){\vector(0,-1){24}}
    \put(31,060){\vector(0,-1){24}}
  \end{picture}
  \end{center}
  \caption{An overview of the \OCCAL\ system}
  \label{fig:overview}
\end{figure}

The symbolic calculation of the matrix entries is of some importance
for the whole computation, since this avoids the (external) numerical
differentiation of $f$ and $r$, which might well lead to problems in
the numerical calculation.  Furthermore, one may reorder the variables
to take advantage of the possible block structure of $R_{1}$ and
$R_{m}$.

Without further manipulation, the resulting code is generated directly
from the expressions obtained in the previous phase, i.e.\ in a fully
expanded form. This is, of course, non-optimal, except for really
simple cases.  Therefore several different optimization strategies can
be applied.

We developed a very simple one that separates the calculation of
algebraic and transcendental functions like $\sqrt {x}$ or ${\mathrm
e}^{x}$, (so that their values are computed only once) and uses
Horner's scheme to evaluate polynomials.  In the case of sparse
polynomials in many variables, however, using Horner's scheme is not
straightforward, since it is not a priori clear how the variables
should be ordered. 

A more elaborate optimization scheme was developed and implemented by
{\sc van Hulzen et al.}\ \cite{scope}. It attempts to minimize the
number of arithmetic operations. The optimizer performs heuristic
searches on arithmetic expressions, detecting and extracting common
subexpressions, and replacing them by temporary variable names.

The choice of optimization strategy depends mainly on the problem at
hand. Complete optimization takes a considerable amount of computing
time, but this may be payed off by the generation of more efficient
and better readable numerical code. For every new problem, it is
therefore necessary to exploit all possibilities to arrive at an
optimal solution.

If, on the other hand, we have a problem with linear control $u^{(l)}$,
the next step is to try to solve the remaining linear optimization
problem. The present \OCCAL{} system is not yet capable to distinguish
between the singular and the bang-bang case. This is being worked on.

Figure~\ref{fig:overview} depicts the overall structure of the system.


\section{Examples}

\label{sec:examples}

In this section we present a few simple and well-known examples to
illustrate the performance of our system,

\vspace*{3mm}\noindent
{\bf Example 1: Turbo Generator \cite{Wulkow}}.

This example is simple enough to do all necessary calculations by
hand, and therefore well-suited to check the performance of \OCCAL{}.

We will not discuss the technical background of the turbo generator as
a machine. Rather we concentrate on the mathematical model derived
from these technical considerations.

The dynamic model of the turbo generator consists of the following system of
ordinary differential equations:
\begin{eqnarray}
  \dot x_{1} &=& x_{2} x_{4} \nonumber \\
  \dot x_{2} &=& \frac{1}{M} (u_{1} - s_{4} x_{1} x_{4} - s_{5} x_{1}
                  x_{3} - \kappa_{d} x_{2}) \nonumber \\
  \dot x_{3} &=& u_{2} - A x_{3} + c x_{4} \nonumber \\
  \dot x_{4} &=& - x_{1} x_{2}
\end{eqnarray}
Here $x_{i}$ are the dynamic variables, $u_{i}$ the control variables,
$s_{4}$, $s_{5}$, $\kappa_{d}$, $A$, $M$, and $c$ are constants.  The
stationary state of the machine is characterized by $x_{i} =
x_{i}^{s}$ and $u_{i}=u_{i}^{s}$, with certain constants $x_{i}^{s}$,
$u_{i}^{s}$.

The functional to be minimized describes the deviation from a
stationary state
\begin{equation}
  \begin{split}
  J_{1}(x,u) = \int\limits_{0}^{T} 
                (& \alpha ((x_{1} - x_{1}^{s})^{2} +
                          (x_{4} - x_{4}^{s})^{2}) +
                  \alpha_{2} x_{2}^{2} +
                  \alpha_{3} (x_{3} - x_{3}^{s})^{2} + \\
              &    \beta_{1} (u_{1} - u_{1}^{s})^{2} +
                  \beta_{2} (u_{2} - u_{2}^{s})^{2}) dt
  \end{split}
\end{equation}
with certain weight factors $\alpha$, $\alpha_{2}$, $\alpha_{3}$,
$\beta_{1}$, and $\beta_{2}$.  Boundary conditions are 
\begin{eqnarray}
  x(0) &=& (x_{1}^{s}, x_{2}^{s}, x_{3}^{s}, x_{4}^{s}) =: x_{0}^{t} \\
  x(T) && \mbox{free}
\end{eqnarray}
The control variables $u^{(n)} = (u_{1},u_{2})$ appear quadratically in
the Hamiltonian, which leads to a system of two linear equations with
a unique solution. After substituting the solution into the
differential equations we end up with a boundary value problem with
four dynamical variables. Due to its simplicity, no optimization of
the code is necessary.

A corresponding \OCCAL{} sample session, the input file for \OCCAL{}
itself, the \REDUCE{} input file generated from this by the
pre-processor, and the \C{} source file are shown in \ref{java.app-turbo}.
It can be seen that the intermediate \REDUCE{} input file is
well-readable, so that the experienced user may want to edit this file
directly. This example takes about $25$ seconds of real time on a
typical (Sun 4) workstation.

The file of generated \C{} code shows several interesting points:
\begin{itemize}
  \item The constants defined in the CONSTANTS section of the input
        file are rendered as \C{} pre-processor macros.

  \item There are three \C{} functions defined: for the boundary
        values (\verb|R1|), for the functions $f_{i}$ (\verb|FCN1|),
        and for the functions $f_{y}$ that are needed in the solution
        of the variational equation (\verb|DFCN1|).

        The functions for the calculation of the matrices $R_{1}$ and
        $R_{m}$ have been skipped in this example.
  \item The code optimization process introduces several intermediate
        variables to reduce the number of operations in the program.
        These are easily identified by their name which consists
        always of the letter \verb|G| (short for generated) followed
        by a number. It is possible to choose other names as well.
\end{itemize}

With a rather simple main program to drive the calculation it is easy
to reproduce the numerical results of \cite{Wulkow}.

\vspace*{3mm}

\noindent
{\bf Example 2: Re-entry problem \cite{SB}.}


This is a well-known test problem of somewhat higher degree of complexity.
Here, the expressions arising from the calculation turn out
to be rather lengthy.

The re-entry of an Apollo vehicle into the Earth's atmosphere is
governed by the following system of differential equations:
\begin{eqnarray}
  \dot v &=& V(v,\gamma,\xi,u) = - \frac{S \rho v^{2}}{2m} C_{W}(u) -
             \frac{g \sin \gamma}{(1+\xi)^{2}}\\
  \dot \gamma &=& \Gamma(v,\gamma,\xi,u) = \frac{S \rho v}{2m}
             C_{A}(u) + \frac {v \cos \gamma}{R(1+\xi)}
             - \frac{g \cos \gamma}{v(1+\xi)^{2}}\\
  \dot \xi &=& \Xi(v,\gamma,\xi,u) = \frac{v\sin\gamma}{R} \\
  \dot \zeta &=& Z(v,\gamma,\xi,u) = \frac{v \cos \gamma}{1+xi}
\end{eqnarray}
where $u$ is the control variable; and state variables are: $v$
(tangential velocity), $\gamma$ (flight path angle), $\xi = h/R$
(relative height over ground), $\zeta$ (velocity over ground).
Moreover, the expressions $\rho = \rho_{0} \exp(-\beta R \xi)$ (),
$C_{W}(u)$ (aerodynamical resistance), $C_{A}(u)$ (drag or lift
coefficient), $S$ (front area), and $m$ (mass of the vehicle) appear
in the above equations..

The boundary values at the begin of the reentry into the atmosphere
and at the time of landing are
\begin{alignat}{3}
 v(0) &= 0.36,& \quad
 \gamma(0) &= -8.1^{\circ} \frac{\pi}{180^{\circ}},& \quad
 \xi(0) &= \frac{4}{R}\\
 v(T) &= 0.27,& \quad
 \gamma(T) &= 0,& \quad
 \xi(T) &= \frac{2.5}{R}
\end{alignat}
The final time $T$ is free; $\zeta$ is not considered for the
optimization. The functional to be minimized is the heating of the
capsule, approximated by
\begin{equation}
  J = \int_{0}^{T} 10\, v^{3} \sqrt{\rho} dt \:.
\end{equation}
The problem has only one control variable, but equation
(\ref{sys-nonlin}) turns out to be transcendental, of the type
(\ref{sin-cos}) discussed earlier. It is possible, by a
change of variables, to turn it into a system of algebraic equations
which can be solved by computing the Groebner base of the polynomials.
This is done automatically by the \REDUCE{} equation solver. As a
result, we get two solutions, only one of which leads to a positive
definite determinant. This can be determined automatically if the
additional information of the variable $v > 0$ is given to the system.
Otherwise the user is prompted for the sign of $v$.

The further calculation then proceeds, in principle, just as in the
previous example. However, the expressions to be turned into numerical
code are quite large, so that we apply optimization techniques during
the code generation process. Of the two levels described above, the
\SCOPE{} package by van Hulzen \cite{scope} is able to reduce the size
of the \C{} source code from $35$ kBytes ($808$ lines) to $12$ kBytes
($378$ short lines). The simpler package which only extracts simple
common subexpression reduces the size to $27$ kBytes ($570$ lines).
Using the latter, the calculation takes about $50$ seconds of real time
on a Sun 4 workstation.

% Two examples including singular control
% 
% \cite{chemo,zellen} are two examples of optimal control problems with
% non-regular control. The first of these is completely solvable
% analytically in the absence of state constraints \cite{chemo}.
% \OCCAL{} is not yet capable to do this completely since this solution
% relies on heuristic methods which are very difficult to make into an
% algorithm. Nevertheless, \OCCAL{} goes very far and is able to
% correctly determine the singular control.
% 
% The second example shows other difficulties: due to the presence of a
% lot of inequalities the calculation of the singular control can be
% simplified significantly. However, dealing with inequalities is a weak
% point in most Computer Algebra Systems (see also next section).



\section{Conclusion}

As documented above, \OCCAL{} has reached a first stage of
development where it can be successfully used to help in the solution
of optimal control problems. The tasks already accomplished are:
\begin{itemize}
  \item Calculation of Hamiltonian.
  \item Determination of nonlinear and linear control.
  \item Code generation with optional optimization.
\end{itemize}
Next steps of development will include the following features that we
are planning to incorporate in the future:
\begin{itemize}
  \item Symbolic-numeric determination of nonlinear control: It is rather obvious
        that not for all systems of equations an analytic solution can
        be found. In these cases the algebraic equations have to be
        coupled to the dynamical system (\ref{dyn-sys}) and the whole
        must be solved numerically.

        The numeric treatment of algebro-differential equations is
        being worked on.

  \item Determination of inequalities: The determination of the sign
        of the determinant (\ref{sys-nonlin}) is only a first simple
        step in the treatment of inequalities. There are a number of
        other places where inequalities arise, e.g., when inequalities
        involving the state variables appear. This is especially
        important in order to find the intervals where the switching
        funtions vanish identically. 

  \item Inclusion of state constraints: these may be very important as
        they may determine the structure of the linear control (i.e.,
        singular vs.\ bang-bang). As far as inequalities are
        concerned, this is related to the previous point, and has
        probably to be treated in a common framework.

  \item Treatment of singular sub-arcs.
\end{itemize}



\noindent
{\bf Acknowledgments.} The authors wish to thank M.~Wulkow for several
helpful discussions.



\begin{thebibliography}{99}

  \bibitem{Bock1} H.~G.~Bock:
  \newblock Randwertproblemmethoden zur Parameteridentifizierung
            in Systemen nichtlinearer Differentialgleichungen,
  \newblock Universit\"at Bonn, Dissertation (1985).

  \bibitem{Bock2} H.~G.~Bock:
  \newblock Numerical Solution of Nonlineaer Multipoint Boundary Value
            Problems with Applications to Optimal Control,
  \newblock GAMM, Copenhagen 1977.

  \bibitem{Bock3} H.~G.~Bock:
  \newblock Numerische Berechnung zustandsbeschr\"ankter optimaler
            Steuerungen mit der Mehrzielmethode,
  \newblock Carl-Cranz-Gesellschaft 1978.

  \bibitem{bulirsch} R.~Bulirsch:
  \newblock Variationsrechnung und optimale Steuerung.
  \newblock Lectures given at the Universit\"at zu K\"oln (1971).
  \newblock Unpublished.

  \bibitem{th1} R.~Bulirsch et al.~(eds.):
  \newblock Optimization and Optimal Control,
  \newblock LN Math.~477 (1975).

  \bibitem{th2} A.~E.~Bryson and Y.~C.~Ho:
  \newblock Applied optimal control,
  \newblock Ginn and Company,
  \newblock Waltham, Massachusetts (1969).

  \bibitem{mulcon} P.~Deuflhard, B.~Engquist (Ed.):
  \newblock Large Scale Scientific Computing,
  \newblock Birkhauser/Boston, Series ``Progress In Scientific Computing'',
  \newblock Vol.~7 (1987).

  \bibitem{DFK2} P.~Deuflhard, B.~Fiedler, P.~Kunkel:
  \newblock Efficient Numerical Pathfollowing Beyond Critical Points
            in ODE models,
  \newblock in: \cite{mulcon}, pp.~97--113 (1987).

  \bibitem{scope} J.~A.~van~Hulzen:
  \newblock Code optimization of multivariate polynomial schemes:
            A pragmatic approach,
  \newblock Proceedings EUROCAL '83,
  \newblock Springer LNCS 162, pp.~286--300.

  \bibitem{th3} H.~Maurer, U.~Heidemann:
  \newblock Optimale Steuerprozesse mit Zustandsbeschr\"ankungen,
  \newblock in: \cite{th1}.

  \bibitem{SB} J.~Stoer, R.~Bulirsch: 
  \newblock Einf\"uhrung in die Numerische Mathematik II,
  \newblock Springer-Verlag,
  \newblock Berlin 1978,
  \newblock ISBN 3--540--08840--7.

  \bibitem{perl} L.~Wall, R.~L.~Schwartz:
  \newblock Programming \perl,
  \newblock O'Reilly \& Associates, Inc.,
  \newblock 1990.

  \bibitem{Wulkow} M.~Wulkow:
  \newblock Vergleich numerischer Verfahren zur Berechnung der
  optimalen Steuerung eines Turbogenerators,
  \newblock Diploma Thesis, Univ.\ M\"unster, 1987.

%  \bibitem{f2c} 

\end{thebibliography}

\newpage

\appendix

\renewcommand{\thesection}{Appendix~\Alph{section}}
\renewcommand{\thesubsection}{\Alph{section}.\arabic{subsection}}

\section[\protect\hspace*{\dummy}An \OCCAL{} sample session\protect\hspace*{-\dummy}]%
        {An \OCCAL{} sample session}

\label{java.app-turbo}
As an example we treat the example of the turbo generator. We assume
that the following input file has been prepared.\\
\rule{\textwidth}{0.6mm}
\begin{small}
  \verbatiminput{turbo.dsc}
\end{small}
\rule{\textwidth}{0.6mm}\\[3mm]
It is further assumed that the name of this file ends in
\verb|.dsc|, e.g., \verb|turbo.dsc|.
Type in:
\begin{verbatim}
occal turbo
\end{verbatim}
\OCCAL{} displays
\begin{verbatim}
Parsing input file...done.
Starting symbolic processor...
REDUCE 3.4, 15-Jul-91 ...

1:

\end{verbatim}
and then a protocol of the following \REDUCE{} statements that were
generated by the pre-processor.\\
\rule{\textwidth}{0.6mm}
\begin{small}
  \verbatiminput{turbo.red}
\end{small}
\rule{\textwidth}{0.6mm}\\[3mm]
After a while it displays
\begin{verbatim}
============================================================
 That's all, folks!
============================================================
\end{verbatim}
and then
\begin{verbatim}
Quitting
OCCAL finished.
\end{verbatim}
There is now a file with name \verb|turbo1.c|. We show its contents for illustration.\\
\rule{\textwidth}{0.6mm}
\begin{small}
  \verbatiminput{turbo1.c}
\end{small}
\rule{\textwidth}{0.6mm}

\newpage

\section[\protect\hspace*{\dummy}Files for the reentry problem]%
        {Files for the reentry problem}

\label{java.app-SB}

\small

\subsection{Problem description}
\verbatiminput{SB.dsc}


\subsection{\REDUCE{} input file}
\verbatiminput{SB.red}

\newpage

\section[\protect\hspace*{\dummy}Input file format for the pre-processor]%
        {Input file format\\ for the pre-processor}

\label{java.app:perl-file}

The input file consists of a number of sections, each identified by a
keyword on a line of its own, and ended by an empty line. Possible
keywords are:
\begin{description}
  \item [COMMENT] Used only to explain what is going on. Everything up
        to the next empty line is ignored.
  \item [NAME] A short name to identify the problem, used for file
        names, etc.
  \item [TITLE] Description of the problem, can consist of several
        lines of text.
  \item [CONSTANTS] Constants of the problem, i.e., identifiers whose
        values do not change. Can only be set at compile time. A value
        is mandatory.
  \item [PARAMETERS] Parameters whose values may be set at run time.
        A value is optional.
  \item [DYNAMIC\_VARIABLES] The dynamic variables, called $y_{i}$ in a
        previous section, each on a line of its own.
  \item [CONTROL\_VARIABLES] The control variables, called $u_{i}$ in
        a previous section, each on a line of its own.
  \item [TIME\_VARIABLE] The name of the independent variable.
  \item [BOUNDARIES] Two expressions in \REDUCE{} syntax that denote
        the boundaries of the interval. Must be delimited by a
        semicolon.
  \item [FUNCTIONAL] The functional to be minimized.
  \item [DIFFERENTIAL\_EQUATIONS] The dynamical system.
  \item [END\_POINT\_IS\_FREE] Specifies that the right boundary is free
        and to be calculated.
  \item [BOUNDARY\_CONDITIONS] The boundary conditions at the points
        specified in the BOUNDARIES section.
  \item [ABBREVIATIONS] Arbitrary \REDUCE{} expressions, each
        delimited by a semicolon. 
  \item [RESTRICTIONS] Additional conditions, such as a variable being
        bounded from below.
  \item [TARGET\_LANGUAGE] The programming language in which the numerical
        programs are generated. Possible are: \C, \FORTRAN, \RATFOR, and \Pascal.
  \item [TEMPVAR\_PREFIX] The prefix used for constructing variable names during
        code generation. Default is \verb|ANS|.
\end{description}
The order of these sections is largely arbitrary, except for one
restriction: the dynamic variables need to be defined before the
corresponding differential equations.


\end{document}
